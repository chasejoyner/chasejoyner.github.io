\documentclass[10pt]{article}
\usepackage{amsmath,amssymb,bm,fullpage,relsize}
\usepackage{multicol}
\usepackage{geometry}
\geometry{
	top = 1.5cm,
	left = 1.75cm,
	right = 1.75cm
}

\newcommand{\x}{\bm{x}}
\renewcommand{\a}{\bm{a}}
\renewcommand{\b}{\bm{b}}
\renewcommand{\u}{\bm{u}}
\renewcommand{\v}{\bm{v}}
\newcommand{\R}{{\mathbb R}}
\newcommand{\zero}{\bm{0}}

\begin{document}

\begin{center}
{\Large \textbf{Chase Joyner}}
\end{center}
O-110 Martin Hall, Box 340975 \hfill chasej@g.clemson.edu \\
Clemson, SC 29634 \hfill (704) 830-4730 \\
\rule{18.08cm}{0.20mm} \\ \\
\noindent
\textbf{Education}
\vspace{-0.25cm}
\begin{itemize}
\item[] Ph.D. in Mathematical Sciences (Statistics) (4.0/4.0) \hfill May 2016 -- Present \\
Clemson University, Clemson, SC
\item[] M.S. in Mathematical Sciences (Statistics) (4.0/4.0) \hfill Aug 2014 -- May 2016 \\
Clemson University, Clemson, SC
\item[] B.S. in Mathematical Sciences (Statistics) (3.89/4.0) \hfill Aug 2010 -- 2014 \\
Clemson University, Clemson, SC
\end{itemize}

\noindent
\textbf{Research Experience: Clemson University}
\vspace{-0.25cm}
\begin{itemize}
\item[] Research Assistant sponsored by Biorealm, Principal Investigator \hfill Jan 2016 -- Dec 2016
\vspace{-0.25cm}
\begin{itemize}
\item[-] Analyzed rice data provided from fields in Indonesia to develop a random effects \\ model accounting for complex genetic similarity.
\vspace{-0.1cm}
\item[-] Modeled the rice production and resistance to climate change in Indonesia.
\vspace{-0.1cm}
\item[-] Simulated data in R using Clemson's cluster to validate the model.
\vspace{-0.1cm}
\item[-] Techniques used: Expectation-Maximization algorithm, generalized linear models, \\ and random effects model.
\vspace{-0.25cm}
\end{itemize}
\item[] Master's Thesis \hfill Aug 2014 -- May 2016
\vspace{-0.25cm}
\begin{itemize}
\item[-] Developed univariate and multivariate Bayesian models to estimate the optimal biomarker \\ density threshold in pooled testing of individuals for various diseases.
\vspace{-0.1cm}
\item[-] Implemented algorithms in R to estimate the parameters of these Bayesian models.
\vspace{-0.1cm}
\item[-] Techniques used: Gibbs sampling, Metropolis-Hastings, and Bayesian iteratively \\ reweighted least squares.
\vspace{-0.25cm}
\end{itemize}
\item[] Undergraduate Thesis \hfill Aug 2013 -- Aug 2014
\vspace{-0.25cm}
\begin{itemize}
\item[-] Analyzed Bayesian techniques and Markov chain Monte Carlo methods for inference.
\vspace{-0.1cm}
\item[-] Documented the implementation of these methods and ran simulations.
\end{itemize}
\end{itemize}

\noindent
\textbf{Research Presentations}
\vspace{-0.25cm}
\begin{itemize}
\item[] Joyner, C. Assessing the relationship between SNPs and yield in various rice varieties. \\
\emph{Jakarta, Indonesia} (Nov 2016).
\vspace{-0.25cm}
\item[] Joyner, C. Bayesian approach of biomarker density estimation using pooled data. \\
\emph{Clemson University} (Feb 2016).
\end{itemize}

\noindent
\textbf{Teaching Experience: Clemson University}
\vspace{-0.25cm}
\begin{itemize}
\item[] Graduate Teacher of Record, MATH 1070: Differential and Integral Calculus \hfill Spring 2017
\vspace{-0.25cm}
\item[] Graduate Teacher of Record, MATH 1040: Precalculus and Introductory Differential Calculus \hfill Fall 2016
\vspace{-0.25cm}
\item[] Graduate Teacher of Record, MATH 1020: Introduction to Mathematical Analysis \hfill Fall 2015
\vspace{-0.25cm}
\item[] Graduate Teaching Assistant \hfill Fall 2014 - Summer 2015
\vspace{-0.25cm}
\begin{itemize}
\item[-] Courses: Statistics for Science and Engineering (MATH 3020), Calculus of One \\ Variable II (MATH 1080), and Differential and Integral Calculus (MATH 1070).
\end{itemize}
\end{itemize}

\noindent
\textbf{Professional Memberships}
\vspace{-0.25cm}
\begin{itemize}
\item[] American Mathematical Society (AMS)
\end{itemize}

\noindent
\textbf{References}
\vspace{-0.25cm}
\begin{itemize}
\item[] Christopher McMahan, Clemson University. Contact: \emph{mcmaha2@g.clemson.edu}.
\vspace{-0.25cm}
\item[] Robert Lund, Clemson University. Contact: \emph{lund@g.clemson.edu}.
\end{itemize}

\end{document}
