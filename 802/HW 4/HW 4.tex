\documentclass[11pt]{article}
\usepackage{amsmath,amssymb,bm,fullpage,relsize}

\newcommand{\x}{\bm{x}}
\renewcommand{\a}{\bm{a}}
\renewcommand{\b}{\bm{b}}
\renewcommand{\u}{\bm{u}}
\renewcommand{\v}{\bm{v}}
\newcommand{\R}{{\mathbb R}}
\newcommand{\zero}{\bm{0}}
\newcommand{\XX}{\mathbf{X}^T\mathbf{X}}

\title{Chase Joyner}
\author{802 Homework 4}
\date{April 3, 2017}

\begin{document}
\maketitle

\section*{Problem 11.2}
As used in the proof to Theorem 11.2b, show that
\[
\int_0^\infty t^ae^{-bt}dt = b^{-(a+1)}\Gamma(a+1).
\]
\begin{itemize}
\item[] \textbf{Solution:}  Let $u = bt$.  Then, by this change of variable we have
\begin{align*}
\int_0^\infty t^ae^{-bt}dt &= \frac{1}{b}\int_0^\infty \left(\frac{u}{b}\right)^ae^{-u}du = b^{-(a+1)}\int_0^\infty u^{a+1-1}e^{-u}du = b^{-(a+1)}\Gamma(a+1),
\end{align*}
where we obtain the last equality using the known Gamma function.
\end{itemize}

\section*{Problem 11.3}
\begin{itemize}
\item[(a)] Show that $(\textbf{I} + \textbf{X}\textbf{V}\textbf{X}')^{-1} = \textbf{I} - \textbf{X}(\textbf{X}'\textbf{X} + \textbf{V}^{-1})^{-1}\textbf{X}'.$
\begin{itemize}
\item[] \textbf{Solution:}  Notice that
\begin{align*}
&(\textbf{I} + \textbf{X}\textbf{V}\textbf{X}')\times (\textbf{I} - \textbf{X}(\textbf{X}'\textbf{X} + \textbf{V}^{-1})^{-1}\textbf{X}') \\
&= \textbf{I} - \textbf{X}(\textbf{X}'\textbf{X} + \textbf{V}^{-1})^{-1}\textbf{X}' + \textbf{X}\textbf{V}\textbf{X}' - \textbf{X}\textbf{V}\textbf{X}'\textbf{X}(\textbf{X}'\textbf{X} + \textbf{V}^{-1})^{-1}\textbf{X}' \\
 &= \textbf{I} + \mathbf{X}\mathbf{V}\left[\textbf{V}^{-1}(\textbf{X}'\textbf{X} + \textbf{V}^{-1})^{-1} + \textbf{I} -  \textbf{X}'\textbf{X}(\textbf{X}'\textbf{X} + \textbf{V}^{-1})^{-1}\right] \mathbf{X}' \\
 &= \textbf{I} + \mathbf{X}\mathbf{V}\left[\textbf{I} - (\mathbf{V}^{-1} + \mathbf{X}'\mathbf{X})(\textbf{X}'\textbf{X} + \textbf{V}^{-1})^{-1}\right] \mathbf{X}' \\
 &= \mathbf{I} + \mathbf{X}\textbf{V}[\mathbf{I}-\textbf{I}]\mathbf{X}' = \mathbf{I}.
\end{align*}
The exact argument above can also be used to show that
\[
(\textbf{I} - \textbf{X}(\textbf{X}'\textbf{X} + \textbf{V}^{-1})^{-1}\textbf{X}')\times (\textbf{I} + \textbf{X}\textbf{V}\textbf{X}') = \mathbf{I}.
\]
This shows the result.
\end{itemize}

\newpage
\item[(b)] Show that $(\textbf{I} + \textbf{X}\textbf{V}\textbf{X}')^{-1}\textbf{X} = \textbf{X}(\textbf{X}'\textbf{X} + \textbf{V}^{-1})^{-1}\textbf{V}^{-1}.$
\begin{itemize}
\item[] \textbf{Solution:}  Considering the difference and part (a), we find
\begin{align*}
&(\textbf{I} + \textbf{X}\textbf{V}\textbf{X}')^{-1}\textbf{X} - \textbf{X}(\textbf{X}'\textbf{X} + \textbf{V}^{-1})^{-1}\textbf{V}^{-1} \\
&= \left[\textbf{I} - \textbf{X}(\textbf{X}'\textbf{X} + \textbf{V}^{-1})^{-1}\textbf{X}'\right]\mathbf{X} - \textbf{X}(\textbf{X}'\textbf{X} + \textbf{V}^{-1})^{-1}\textbf{V}^{-1} \\
&= \textbf{X} - \textbf{X}(\textbf{X}'\textbf{X} + \textbf{V}^{-1})^{-1}\textbf{X}'\textbf{X} - \textbf{X}(\textbf{X}'\textbf{X} + \textbf{V}^{-1})^{-1}\textbf{V}^{-1} \\
&= \textbf{X} - \textbf{X}(\textbf{X}'\textbf{X} + \textbf{V}^{-1})^{-1}(\textbf{X}'\textbf{X} + \textbf{V}^{-1}) \\
&= \textbf{X} - \textbf{X} = \textbf{0}
\end{align*}
which shows the result.%
\end{itemize}

\item[(c)] Show that $\textbf{V}^{-1} - \textbf{V}^{-1}(\textbf{X}'\textbf{X} + \textbf{V}^{-1})^{-1}\textbf{V}^{-1} = \textbf{X}'(\textbf{I} + \textbf{X}\textbf{V}\textbf{X}')^{-1}\textbf{X}$.
\begin{itemize}
\item[] \textbf{Solution:}  By equation (2.54) in the text, we obtain
\begin{align*}
&\textbf{V}^{-1} - \textbf{V}^{-1}(\textbf{X}'\textbf{X} + \textbf{V}^{-1})^{-1}\textbf{V}^{-1} \\
&= \left[\textbf{V} + (\textbf{X}'\textbf{X})^{-1}(\textbf{X}'\textbf{X})(\textbf{X}'\textbf{X})^{-1}\right]^{-1} 
\\
&= \left[(\textbf{X}'\textbf{X})^{-1} + \textbf{V}\right]^{-1}.
\end{align*}
Now, by (2.54) again, 
\[
\left[(\textbf{X}'\textbf{X})^{-1} + \textbf{V}\right]^{-1} = \textbf{X}'\textbf{X} - \textbf{X}'\textbf{X}(\textbf{X}'\textbf{X} + \textbf{V}^{-1})^{-1}\textbf{X}'\textbf{X}
\]
and therefore we have that
\begin{align*}
&\textbf{V}^{-1} - \textbf{V}^{-1}(\textbf{X}'\textbf{X} + \textbf{V}^{-1})^{-1}\textbf{V}^{-1} \\
&= \textbf{X}'\textbf{X} - \textbf{X}'\textbf{X}(\textbf{X}'\textbf{X} + \textbf{V}^{-1})^{-1}\textbf{X}'\textbf{X} \\
&= \textbf{X}'\left[\textbf{I} - \textbf{X}(\textbf{X}'\textbf{X} + \textbf{V}^{-1})^{-1}\textbf{X}'\right]\textbf{X} \\
&= \textbf{X}'(\textbf{I} + \textbf{X}\textbf{V}\textbf{X}')^{-1}\textbf{X}
\end{align*}
where the last equality is from part (a).
\end{itemize}
\end{itemize}

\newpage
\section*{Problem 5.20}
\begin{itemize}
\item[(a)] We can see that the $t$ distribution is a mixture of normals using the following argument:
\[
P(T_\nu \leq t) = P\left(\frac{Z}{\sqrt{\chi^2_\nu / \nu}}\leq t\right) = \int_0^\infty P(Z \leq t\sqrt{x}/\sqrt{\nu})P(\chi^2_\nu = x)dx,
\]
where $T_\nu$ is a $t$ random variable with $\nu$ degrees of freedom.  Using the Fundamental Theorem of Calculus and interpreting $P(\chi^2_\nu = x)$ as a pdf, we obtain
\[
f_{T_\nu}(t) = \int_0^\infty \frac{1}{\sqrt{2\pi}}e^{-t^2x/2\nu}\frac{\sqrt{x}}{\sqrt{\nu}}\frac{1}{\Gamma(\nu/2)2^{\nu/2}}x^{\nu/2-1}e^{-x/2}dx,
\]
a scale mixture of normals.  Verify this formula by direct integration.
\begin{itemize}
\item[] \textbf{Solution:}  Notice that
\begin{align*}
&\int_0^\infty \frac{1}{\sqrt{2\pi}}e^{-t^2x/2\nu}\frac{\sqrt{x}}{\sqrt{\nu}}\frac{1}{\Gamma(\nu/2)2^{\nu/2}}x^{\nu/2+1/2-1}e^{-x/2}dx \\
&= \frac{1}{\sqrt{2\pi}}\frac{1}{\sqrt{\nu}\hspace{1mm}\Gamma(\nu/2)2^{\nu/2}}\int_0^\infty x^{(\nu+1)/2-1}e^{ -t^2x/2\nu - x/2}dx \\
&= \frac{1}{\sqrt{2\pi}}\frac{1}{\sqrt{\nu}\hspace{1mm}\Gamma(\nu/2)2^{\nu/2}}\int_0^\infty x^{(\nu+1)/2-1}e^{ -x(t^2/2\nu + 1/2)}dx \\
&= \frac{1}{\sqrt{2\pi}}\frac{1}{\sqrt{\nu}\hspace{1mm}\Gamma((\nu/2)2^{\nu/2}}\cdot  \Gamma\big{(}(\nu+1)/2\big{)}\frac{1}{(t^2/2\nu + 1/2)^{(\nu+1)/2}} \\
&= \frac{1}{\sqrt{2\pi}}\frac{ \Gamma\big{(}(\nu+1)/2\big{)}}{\sqrt{\nu}\hspace{1mm}\Gamma((\nu/2)2^{\nu/2}} \left(\frac{2\nu}{t^2+\nu}\right)^{(\nu+1)/2} \\
&= \frac{1}{\sqrt{\nu\pi}}\frac{ \Gamma\big{(}(\nu+1)/2\big{)}}{\Gamma((\nu/2)2^{\nu/2}} \left(\frac{1}{t^2+\nu}\right)^{(\nu+1)/2}
\end{align*}
which is the pdf of a $t_\nu$ random variable.  This shows the result.
\end{itemize}

\item[(b)] A similar formula holds for the $F$ distribution; that is, it can be written as a mixture of chi squareds.  If $F_{1,\nu}$ is an $F$ random variable with 1 and $\nu$ degrees of freedom, then we can write
\[
P(F_{1,\nu} \leq \nu t) = \int_0^\infty P(\chi^2_1\leq ty)f_\nu(y)dy,
\]
where $f_\nu(y)$ is a $\chi^2_\nu$ pdf.  Use the Fundamental Theorem of Calculus to obtain an integral expression for the pdf of $F_{1,\nu}$, and show that the integral equals the pdf.
\begin{itemize}
\item[] \textbf{Solution:}  Taking the derivative of both sides and interchanging the derivative with the integral, we have
\begin{align*}
\nu f_{1,\nu}(\nu t) &= \frac{d}{dt} \int_0^\infty P(\chi^2_1\leq ty)f_\nu(y)dy =  \int_0^\infty \frac{d}{dt}
P(\chi^2_1\leq ty)f_\nu(y)dy \\
&= \int_0^\infty yf_1(ty)f_\nu(y)dy \\
&= \int_0^\infty y \frac{1}{\sqrt{2}\hspace{1mm}\Gamma(1/2)} (ty)^{-1/2}e^{-ty/2} \cdot \frac{1}{2^{\nu/2}\Gamma(\nu/2)}y^{\nu/2-1}e^{-y/2}dy \\
&= \frac{t^{-1/2}}{2^{(\nu+1)/2}\Gamma(1/2)\Gamma(\nu/2)} \int_0^\infty y^{(\nu+1)/2 - 1}e^{-(t+1)y/2}dy \\
&= \frac{t^{-1/2}}{2^{(\nu+1)/2}\Gamma(1/2)\Gamma(\nu/2)}\cdot \frac{\Gamma\big{(}(\nu+1)/2\big{)}}{\big{[}(t+1)/2\big{]}^{(\nu+1)/2}} \\
&= \frac{t^{-1/2}}{\Gamma(1/2)\Gamma(\nu/2)}\cdot\frac{\Gamma\big{(}(\nu+1)/2\big{)}}{(t+1)^{(\nu+1)/2}}.
\end{align*}
This implies that 
\[
f_{1,\nu}(\nu t) = \frac{t^{-1/2}}{\nu\Gamma(1/2)\Gamma(\nu/2)}\cdot\frac{\Gamma\big{(}(\nu+1)/2\big{)}}{(t+1)^{(\nu+1)/2}}.
\]
For clarity, consider $x = \nu t$ and so the above becomes
\begin{align*}
f_{1,\nu}(\nu t) &= \frac{\Gamma\big{(}(\nu+1)/2\big{)}}{\nu\Gamma(1/2)\Gamma(\nu/2)}\frac{(x/\nu)^{-1/2}}{(1+x/\nu)^{(\nu+1)/2}} \\
&= \frac{\Gamma\big{(}(\nu+1)/2\big{)}\nu^{\nu/2}}{\Gamma(1/2)\Gamma(\nu/2)}\frac{x^{1/2 - 1}}{(\nu + x)^{(1+\nu)/2}}
\end{align*}
and the RHS above is indeed the pdf of an $F_{1,\nu}$ random variable.
\end{itemize}
\end{itemize}



\end{document}
